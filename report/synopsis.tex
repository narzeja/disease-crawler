%\documentclass[10pt,a4paper,final,oneside,openany,article]{memoir}
%\documentclass[letterpaper,a4paper,10pt]{article}
\documentclass[10pt,letterpaper,final]{article}
\usepackage[utf8]{inputenc}
\usepackage[british]{babel}
\usepackage{hyperref}
\setcounter{tocdepth}{3}
\usepackage[draft]{fixme}
\usepackage{abstract}

%% FONT
%\usepackage[T1]{fontenc}
%\usepackage{lmodern}
%\usepackage[urw-garamond]{mathdesign}
% Fonts configuration
%  - Palatino and Bitstream Vera Sans Mono for verbatim
\usepackage[T1]{fontenc}
\usepackage{palatino}
\usepackage[sc]{mathpazo} % Math font for Palatino
\usepackage{bera}
\linespread{1.05} % Palatino needs more leading (space between lines)
\usepackage{microtype} % ++


%% odds and ends
%\chapterstyle{hangnum}
\setcounter{secnumdepth}{2}

%HEADINGS
\title{Constructing Models For Diagnosing Rare Diseases}
%        \small{Synopsis}}

\author{Brian S. Mathiasen $-$ soborg@diku.dk \\
        Henrik G. Jensen $-$ henne@diku.dk\\
%        \\
%        Department of Computer Science\\
%        University of Copenhagen\\
%        Universitetsparken 1\\
%        DK-2100 Copenhagen, Denmark
}

\date{\today} %\today

%%
\begin{document}
\maketitle
%\listoffixmes
%\tableofcontents
\section{Disposition}

The use of the Internet for aiding in diagnoses is an important part in
the daily practice of physicians. It is therefore important to develop
systems that will aid physicians, with particular focus on rare
diseases.

The goal of the project is to develop a system for building language
models for rare diseases. The language models will be based on prior
knowledge provided by Orphanet\footnote{\url{http://orpha.net/}}. It
will be expanded through automated iterations on popular search engines.
The information gathered from the results of these searches will be
validated based on a correlation with the prior knowledge and new
searches will be based on the posterior language model. We will use
machine learning and natural language processing to extract the relevant
information with a focus on symptoms and other important features.

During the project, we will evaluate the system by it’s ability to rank
a relevant disease (the precision of the system) compared with previous
similar projects. A ranking is produced by entering a case report or a
list of symptoms of a rare disease into the system.


\section{Learning Goals}
\begin{itemize}
\item Utilize natural language processing techniques.
\item Utilize intermediate and advanced algorithms for machine learning.
\item Utilize efficient data structures for data mining.
\item Build language models for rare diseases based on results from
popular search engines such as Google and PubMed.
\item Build a model for disease/symptom harvesting.
\item Utilize background knowledge of selected rare diseases to
determine the validity and accuracy of the model.
\item Compare validity and accuracy to previous work on support decision
systems on diagnosing rare diseases.
\end{itemize}

\section{Project Time Frame}
\begin{center}
	\begin{tabular}{lll}
		Week & Activity & Subtasks \\ \hline
		1 & Programming & Select data format, arrange prior data \\
		2 & Programming & preliminary test searches, begin model coding \\
		3 & Programming & model and preliminary results \\
		4 & fine-tuning & fine tune programming \\
		5 & define test & . \\
		6 & tests, results & possible fine-tuning. \\
		7 & further test or buffer, report & . \\
		8 & report or buffer & . \\
		9 & report & . \\
	\end{tabular}
	\label{tab:timeestimates}
\end{center}




%\renewcommand\bibname{References}
%\bibliography{bib}
%\bibliographystyle{apalike}

\end{document}
