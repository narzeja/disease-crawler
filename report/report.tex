%\documentclass[10pt,a4paper,final,oneside,openany,article]{memoir}
%\documentclass[letterpaper,a4paper,10pt]{article}
\documentclass[10pt,letterpaper,final]{article}
\usepackage[utf8]{inputenc}
\usepackage[british]{babel}
\usepackage{hyperref}
\setcounter{tocdepth}{3}
\usepackage[draft]{fixme}
\usepackage{abstract}

%% FONT
%\usepackage[T1]{fontenc}
%\usepackage{lmodern}
%\usepackage[urw-garamond]{mathdesign}
% Fonts configuration
%  - Palatino and Bitstream Vera Sans Mono for verbatim
\usepackage[T1]{fontenc}
\usepackage{palatino}
\usepackage[sc]{mathpazo} % Math font for Palatino
\usepackage{bera}
\linespread{1.05} % Palatino needs more leading (space between lines)
\usepackage{microtype} % ++


%% odds and ends
%\chapterstyle{hangnum}
\setcounter{secnumdepth}{2}

%HEADINGS
\title{Constructing Models For Diagnosing Rare Diseases\\
        \small{by Using the Google Search Engine}}

\author{Brian S. Mathiasen $-$ soborg@diku.dk \\
        Henrik G. Jensen $-$ henne@diku.dk\\
%        \\
%        Department of Computer Science\\
%        University of Copenhagen\\
%        Universitetsparken 1\\
%        DK-2100 Copenhagen, Denmark
}

\date{\today} %\today

%%
\begin{document}
\maketitle
\listoffixmes
%\tableofcontents


\begin{abstract}
In this paper we design and construct models for assisting physicians
with the task of diagnosing rare diseases. Using a prior knowledge of
rare diseases consisting of \textit{disease name, some symptoms} and
\textit{synonyms}, we utilize the \textit{Google Search Engine} to build
the model over several iterations. The model is constructed using
several machine learning and natural language processing techniques by
text mining symptoms and features from abstracts and other interesting
paragraphs of text found in the search process.

Our tests show that... blablabla, \fxwarning{Abstract, results: to do}
\end{abstract}
%%-----------------------------
%\newpage
\section{Introduction}
\label{chap:introduction}


The structure of this report is as follows:

\subsection{Limitations}


\section{Previous Work}
Blabla work by radu and paula \cite{radupaula}.


\cite{jensenandersen} worked on a system for diagnosing rare diseases
using vector space models and text mining. They showed a clear potential
in automating such a process, but their results were not statistically
significant. Their system did not rely on prior knowledge of particular
diseases, but still showed that around 60\% of the tests were
satisfactory.


The article \cite{googlingdiagnosis} is the base line for our research.
It proves the viability of using Google as a diagnosis aid, despite the
tests being done manually with little thought done with regard to
automation or test reliability and validity.
%Our system will move on to the next step and introduce intelligent
%automation through a number of predefined criteria and utilizing several
%machine learning and natural language processing techniques.


The editorials \cite{googlechangemedicine} and
\cite{diagnosissearchengines} suggest Google and search engines in
general as being a very strong player in the task of diagnosing
diseases. The witness of search engines being a valuable source of
information is apparent, but the shear amount of misinformation and
noise from regular searches make such a system in it’s current
incarnation unreliable and error prone.
%(Our goal is to refine and provide means for filtering out this noise
%and focus on harvesting information from related material with reference
%to the original search terms as well as specific criteria for the text
%mining process.)


Inspiration for a system: Online Search Engine specialized for
diagnosing stuff etc.
http://www.healthline.com/symptomsearch

This system is based on simple searches that relies on an already
established database of diseases with appropriate meta-information. The
user is able to search for symptoms and can narrow down the search by
adding several symptoms. The system will then present a list of possible
diseases according to some relevance criterion.

%(It is an end-user focused system, while our system will be a back-bone
%focused system, attempting to establish a model on which queries may be
%imposed. It will as such automate the information collection based on
%several machine learning and natural language processing techniques
%appropriate for the needs.)


Google Health: http://www.google.com/intl/en-US/health/about/index.html


Article by practicing Doctor on “do’s and don’t’s when using Google as
diagnosis aid”:
http://vitualis.wordpress.com/2007/02/26/google-based-medicine/


\section{Method}
We will use Natural Language Processing (NLP) techniques to identify
whether a term or phrase is a symptom. E.g. \texttt{thickened blood} is
a symptom (phrase) while the terms \texttt{thickened} and \texttt{blood}
on their own are not.

Our model is based on Machine Learning (ML) clustering algorithms. The
model, that is trained to classify symptoms into clusters, recognises
that the majority of those symptoms belong to the cluster of data that
is associated with one particular disease. This results in a possible
diagnosis of the disease for the patient, given some level of confidence
or error.

Each cluster within the set of clusters $c \in C$ represents a disease
which has it's own symptoms that is collected through a prior data-set,
as provided by OrphanetData \fxfatal{præciser kilden}, and subsequently
collected through various iterations on search engines and harvested
using NLP techniques for feature extraction.

The process is outline as shown below:
\begin{itemize}
\item extract features from known abstract
\item initiate $n$ iterations to harvest more features based on already
known features and disease meta-data.
\end{itemize}


\subsection{Feature Extraction}
we do this and that and those... innit.


\subsubsection{Feature Extraction as a Classification Problem}
When extracting features, we really wish to be able to distinguish keyword or
feature phrases and words of interest from ordinary phrases and words. Looking
upon the problem as a classification problem allows us to utilize algorithms for
(un-)supervised classification, with the supervised allowing us to directly
measure the root mean square error (RMSE).

We will use NLP techniques to extract candidates for this classification, using
Part-of-Speech (POS) tagging\footnote{\fxwarning{Brief explanation here}}.

%As such, we wish to cluster symptoms $\Sigma$ into clusters $C$ for
%which any disease $d \in C$ has a list of symptoms $\Sigma$.

%Read more at Suite101: Unsupervised Machine Learning and Diagnostics:
%\url{http://www.suite101.com/content/unsupervised-machine-learning-and-diagnostics-a218637}


\appendix
\section{Appendix}

\renewcommand\bibname{References}
\bibliography{bib}
\bibliographystyle{apalike}

\end{document}
